\documentclass[14pt,a4paper,report]{report}
\usepackage[a4paper, mag=1000, left=2.5cm, right=1cm, top=2cm, bottom=2cm, headsep=0.7cm, footskip=1cm]{geometry}
\usepackage[utf8]{inputenc}
\usepackage[english,russian]{babel}
\usepackage{indentfirst}
\usepackage[dvipsnames]{xcolor}
\usepackage[colorlinks]{hyperref}
\usepackage{listings} 
\usepackage{fancyhdr}
\usepackage{caption}
\usepackage{graphicx}
\hypersetup{
	colorlinks = true,
	linkcolor  = black
}

\usepackage{titlesec}
\titleformat{\chapter}
{\Large\bfseries} % format
{}                % label
{0pt}             % sep
{\huge}           % before-code


\DeclareCaptionFont{white}{\color{white}} 

% Listing description
\usepackage{listings} 
\DeclareCaptionFormat{listing}{\colorbox{gray}{\parbox{\textwidth}{#1#2#3}}}
\captionsetup[lstlisting]{format=listing,labelfont=white,textfont=white}
\lstset{ 
	% Listing settings
	inputencoding = utf8,			
	extendedchars = \true, 
	keepspaces = true, 			  	 % Поддержка кириллицы и пробелов в комментариях
	language = C,            	 	 % Язык программирования (для подсветки)
	basicstyle = \small\sffamily, 	 % Размер и начертание шрифта для подсветки кода
	%numbers = left,               	 % Где поставить нумерацию строк (слева\справа)
	numberstyle = \tiny,          	 % Размер шрифта для номеров строк
	stepnumber = 1,               	 % Размер шага между двумя номерами строк
	numbersep = 5pt,              	 % Как далеко отстоят номера строк от подсвечиваемого кода
	backgroundcolor = \color{white}, % Цвет фона подсветки - используем \usepackage{color}
	showspaces = false,           	 % Показывать или нет пробелы специальными отступами
	showstringspaces = false,    	 % Показывать или нет пробелы в строках
	showtabs = false,           	 % Показывать или нет табуляцию в строках
	%frame = single,              	 % Рисовать рамку вокруг кода
	tabsize = 2,                  	 % Размер табуляции по умолчанию равен 2 пробелам
	captionpos = t,             	 % Позиция заголовка вверху [t] или внизу [b] 
	breaklines = true,           	 % Автоматически переносить строки (да\нет)
	breakatwhitespace = false,   	 % Переносить строки только если есть пробел
	escapeinside = {\%*}{*)}      	 % Если нужно добавить комментарии в коде
}

\begin{document}

\def\contentsname{Содержание}

% Titlepage
\begin{titlepage}
	\begin{center}
		\textsc{Санкт-Петербургский государственный политехнический 
			университет Петра Великого\\[5mm]
			Кафедра компьютерных систем и программных технологий}
		
		\vfill
		
		\textbf{Отчёт по лабораторной работе\\[3mm]
			Курс: Операционные системы»\\[6mm]
			Тема: «Интерпретаторы командной строки Linux»\\[35mm]
		}
	\end{center}
	
	\hfill
	\begin{minipage}{.5\textwidth}
		Выполнил студент:\\[2mm] 
		Бояркин Никита Сергеевич\\
		Группа: 43501/3\\[5mm]
		
		Проверил:\\[2mm] 
		Душутина Елена Владимировна
	\end{minipage}
	\vfill
	\begin{center}
		Санкт-Петербург\\ \the\year\ г.
	\end{center}
\end{titlepage}

% Contents
\tableofcontents
\clearpage

\chapter{Лабораторная работа №1}

\section{Цель работы}

\begin{itemize}
	\item Изучение основных команд пользовательского интерфейса.
	\item Изучение цикла подготовки и исполнения программ.
	\item Изучение команд и утилит обработки текстов.
\end{itemize}

\section{Программа работы}

\begin{itemize}
	\item Основы работы c командным интерфейсом.
	\item Команды работы с файловой системой.
	\item Цикл работы программы (компиляторы, запуск отладчика).
	\item Организация конвейера.
	\item Изучение команд и утилит обработки текстов.
\end{itemize}

\section{Ход работы}

\subsection{Основы работы c командным интерфейсом}

Введем набор команд для получения информации об ОС и текущем сеансе:

\lstinputlisting{listings/2.1.log}

\begin{itemize}
	\item \emph{date} - выводит информацию о текущем системном времени.
	\item \emph{who} - выводит пользователей системы, которые в данный момент находятся в ней.
	\item \emph{whoami} - выводит имя пользователя, ассоциированное с текущим эффективным идентификатором пользователя.
	\item \emph{tty} - выводит на экран полное имя файла-терминала.
	\item \emph{logname} - выводит имя пользователя, под которым он произвел вход в систему.
	\item \emph{uname} - выводит на экран имя UNIX-системы. 
\end{itemize}

\clearpage

Изучим команду задержки на указанное время:

\lstinputlisting{listings/2.2.log}

\emph{sleep} - задерживает на указанное время (задается в секундах, однако можно задавать в часах, например 5h).

Команду \emph{sleep} (как и другие) можно преждевременно остановить, послав сигнал прерывания, с помощью комбинации клавиш \emph{Ctrl+C}.

С помощью команды \emph{man}, можно получить справочную информацию о любой команде в формате справочника:

\lstinputlisting{listings/2.3.log}

Справочники в ОС Linux имеют следующие разделы:

\begin{itemize}
	\item \emph{NAME} - указывается название команды и ее функциональное применение.
	\item \emph{SYNOPSIS} - указывается синтаксис команды (все что не заключено в квадратные скобки обязательно к добавлению).
	\item \emph{DESCRIPTION} - описание флагов программы.
	\item \emph{EXAMPLES} - примеры работы команды.
	\item \emph{AUTHOR} - автор программы.
	\item \emph{REPORTING BUGS} - контактные данные для обращения по поводу выявленных ошибок.
	\item \emph{COPYRIGHT} - информация о лицензии.
	\item \emph{SEE ALSO} - список похожих команд, рекомендованных для просмотра.
\end{itemize}

\subsection{Команды работы с файловой системой}

Изучим команду получения информации о файлах и папках \emph{ls}:

\lstinputlisting{listings/2.4.a.log}

Команда \emph{ls} без ключей и указания абсолютного пути к директории выведет все файлы и папки в текущем каталоге. Если используется ключ -l, то выводится информация в виде таблицы: в первом столбце выводятся права доступа для пользователя, группы и остальных на файлы и каталоги (r - чтение, w - запись, x - выполнение). В следующих столбцах выводится количество ссылок на файл или каталог, имя владельца и имя группы, размер в байтах, дата последней модификации и самого файла или каталога имя. Если задать параметром не директорию, а файл, то тогда команда выведет только название этого файла.

\clearpage

Изучим команду просмотра содержимого файлов \emph{cat}:

\lstinputlisting{listings/2.4.b.log}

Команда \emph{cat} выводит содержимое указанных файлов (одного или нескольких) или печатает сообщение об ошибке, если параметром указана директория.

\clearpage


\lstinputlisting{listings/2.4.c.log}
%\lstinputlisting{listings/2.5.log}
%\lstinputlisting{listings/2.6.log}


\subsection{Цикл работы программы}




\subsection{Организация конвейера}


\subsection{Изучение команд и утилит обработки текстов}















\end{document}