\documentclass[14pt,a4paper,report]{report}
\usepackage[a4paper, mag=1000, left=2.5cm, right=1cm, top=2cm, bottom=2cm, headsep=0.7cm, footskip=1cm]{geometry}
\usepackage[utf8]{inputenc}
\usepackage[english,russian]{babel}
\usepackage{indentfirst}
\usepackage[dvipsnames]{xcolor}
\usepackage[colorlinks]{hyperref}
\usepackage{listings} 
\usepackage{fancyhdr}
\usepackage{caption}
\usepackage{graphicx}
\hypersetup{
	colorlinks = true,
	linkcolor  = black
}

\usepackage{titlesec}
\titleformat{\chapter}
{\Large\bfseries} % format
{}                % label
{0pt}             % sep
{\huge}           % before-code


\DeclareCaptionFont{white}{\color{white}} 

% Listing description
\usepackage{listings} 
\DeclareCaptionFormat{listing}{\colorbox{gray}{\parbox{\textwidth}{#1#2#3}}}
\captionsetup[lstlisting]{format=listing,labelfont=white,textfont=white}
\lstset{ 
	% Listing settings
	inputencoding = utf8,			
	extendedchars = \true, 
	keepspaces = true, 			  	 % Поддержка кириллицы и пробелов в комментариях
	language = C,            	 	 % Язык программирования (для подсветки)
	basicstyle = \small\sffamily, 	 % Размер и начертание шрифта для подсветки кода
	%numbers = left,               	 % Где поставить нумерацию строк (слева\справа)
	numberstyle = \tiny,          	 % Размер шрифта для номеров строк
	stepnumber = 1,               	 % Размер шага между двумя номерами строк
	numbersep = 5pt,              	 % Как далеко отстоят номера строк от подсвечиваемого кода
	backgroundcolor = \color{white}, % Цвет фона подсветки - используем \usepackage{color}
	showspaces = false,           	 % Показывать или нет пробелы специальными отступами
	showstringspaces = false,    	 % Показывать или нет пробелы в строках
	showtabs = false,           	 % Показывать или нет табуляцию в строках
	%frame = single,              	 % Рисовать рамку вокруг кода
	tabsize = 2,                  	 % Размер табуляции по умолчанию равен 2 пробелам
	captionpos = t,             	 % Позиция заголовка вверху [t] или внизу [b] 
	breaklines = true,           	 % Автоматически переносить строки (да\нет)
	breakatwhitespace = false,   	 % Переносить строки только если есть пробел
	escapeinside = {\%*}{*)}      	 % Если нужно добавить комментарии в коде
}

\begin{document}

\def\contentsname{Содержание}

% Titlepage
\begin{titlepage}
	\begin{center}
		\textsc{Санкт-Петербургский политехнический 
			университет Петра Великого\\[5mm]
			Кафедра компьютерных систем и программных технологий}
		
		\vfill
		
		\textbf{Отчёт по лабораторной работе\\[3mm]
			Курс: Операционные системы»\\[6mm]
			Тема: «Файловые системы»\\[35mm]
		}
	\end{center}
	
	\hfill
	\begin{minipage}{.5\textwidth}
		Выполнил студент:\\[2mm] 
		Бояркин Никита Сергеевич\\
		Группа: 43501/3\\[5mm]
		
		Проверил:\\[2mm] 
		Душутина Елена Владимировна
	\end{minipage}
	\vfill
	\begin{center}
		Санкт-Петербург\\ \the\year\ г.
	\end{center}
\end{titlepage}

% Contents
\tableofcontents
\clearpage

\chapter{Лабораторная работа №2}

\section{Характеристики системы}

Некоторая информация об операционной системе и текущем пользователе:

\lstinputlisting{listings/0.log}

\section{Ход работы}

\subsection{Фильтрация по одному примеру каждого типа файла}

\subsubsection{Решение в командной строке}

Разработаем команду, которая выведет по одному примеру каждого типа файла из корневого каталога:

\lstinputlisting{listings/1.log}

Рассмотрим команду подробно:

\begin{itemize}
	\item \emph{ls / -l -R} - устанавливаем рекурсивный поиск по корневому каталогу с выводом полной информации.
	\item \emph{awk} - скрипт, который добавляет полный путь в название файла.
	\item \emph{if (\$0\textasciitilde/\textasciicircum\textbackslash//) path=substr(\$0, 0, length(\$0)-1); } - если строка начинается с / (каталог), то сохраняем текущий путь в переменную.
	\item \emph{else \{ if(\$0\textasciitilde/\textasciicircum1/) \$(NF-2)=path"/"\$(NF-2);} - иначе, если это символьная ссылка (начинается с l), то изменяем путь в столбце (NF-2).
	\item \emph{else \{\$NF=path"/"\$NF\} print \$0\} \} } - иначе заменяем путь в последнем столбце (NF).
	\item \emph{grep -v \textasciicircum/} - избавляемся от вывода каталогов.
	\item \emph{sort -k1.1,1.1} - сортировка по первому символу.
	\item \emph{uniq -w1} - уникальность по первому символу.
\end{itemize}

\clearpage

В результате работы команды были получены типы файлов с префиксами \emph{-, b, c, d, l}, однако есть еще два типа файлов, их префиксы \emph{p, s}. Рассмотрим каждый префикс подробнее:

\begin{itemize}
	\item \emph{-} файл, обеспечивает хранение символьных и двоичных данных.
	\item \emph{b} - блочное устройство, обеспечивает обращение к аппаратному обеспечению компьютера. Пример блочного устройства - жесткий диск.
	\item \emph{c} - символьное устройство, обеспечивает обращение к аппаратному обеспечению компьютера. Пример символьного устройства - терминал.
	\item \emph{d} - каталог, обеспечивает организацию доступа к файлам.
	\item \emph{l} - символьная ссылка, обеспечивает предоставление доступа к файлам, расположенным на любых носителях.
	\item \emph{p} - канал (FIFO), обеспечивает организацию взаимодействия процессов в операционной системе.
	\item \emph{s} - сокет, обеспечивает организацию взаимодействия процессов в операционной системе.
\end{itemize}

\subsubsection{Решение в виде bash скрипта}

Решение аналогично предыдущему пункту, однако оформлено в виде \emph{bash} скрипта. Отличие заключается в получении имени файла из аргументов командной строки и запись решения в этот файл:

\lstinputlisting{listings/1.sh}

Запуск скрипта на исполнение происходит следующим образом:

\begin{verbatim}
nikita@nikita-pc:~/temp1$ sudo sh 1.sh filename
\end{verbatim}

В папке со скриптом создался файл \emph{filename}, в котором находится результат работы скрипта, аналогичный предыдущему пункту. 

\subsection{Получение всех жестких ссылок на файл}

С помощью использования индексного дескриптора найдем все ссылки на указанный файл:

\lstinputlisting{listings/2.sh}

Результаты работы скрипта:

\lstinputlisting{listings/2.log}

\subsection{Анализ всех способов формирования ссылок}

Рассмотрим действия команд \emph{link, ln, ln -s, cp}. С помощью команды \emph{ls -l} выясним какого рода объекты они порождают:

\lstinputlisting{listings/3.log}

Сделаем вывод о назначении команд \emph{link, ln, ln -s, cp}: 

\begin{itemize}
	\item \emph{link} - позволяет создавать только жесткие ссылки.
	\item \emph{ln} - без ключей создает жесткую ссылку на файл.
	\item \emph{ln -s} - с ключем \emph{-s} создает символьную ссылку на файл.
	\item \emph{cp} - создает новый файл.
\end{itemize}

\subsubsection{Вывод всех символьных ссылок на файл}

Напишем скрипт подсчитывающий все символьные ссылки на указанный файл:

\lstinputlisting{listings/3.sh}

Результат работы скрипта:

\lstinputlisting{listings/3.r.log}

\subsection{}

\subsection{Утилита find}

\emph{find} - утилита для поиска файлов по имени и другим свойствам в UNIX-подобных ОС. Может проводить поиск в одном или нескольких каталогах, с использованием критериев, заданных пользователем. По умолчанию возвращает все файлы в рабочей директории. также \emph{find} позволяет применять действия ко всем найденным файлам.

Рассмотрим возможности команды \emph{find} с несколькими ключами:

\lstinputlisting{listings/5.log}

Рассмотрим результат исследования команды с несколькими ключами подробнее:

\begin{itemize}
	\item \emph{find -type} - осуществляет поиск по типу файла.
	\item \emph{find -name} - осуществляет поиск файлов по имени, в основном используется для поиска по маске.
	\item \emph{find -size} - осуществляет поиск файлов по размеру. Можно устанавливать нижнюю границу размера файла, верхнюю или обе вместе.
	\item \emph{find -exec} - позволяет создавать вложенные команды. Аргумент "\{\}" заменяется на имя рассматриваемого файла, каждый раз, когда он встречается среди аргументов команды. Все символы за флагом \emph{-exec} считаются ее аргументами до символа ";".
\end{itemize}

\subsection{Утилиты od и *dump}

Рассмотрим команды \emph{od} и \emph{hexdump} с флагами \emph{-c, -bc}:

\lstinputlisting{listings/6.log}

\end{document}